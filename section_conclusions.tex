\section{Conclusions}
A framework for up-scaling DEM simulations is presented in this article. Up-scaling is achieved by matching homogenized stress-strain curves from REV-scale DEM simulations to single element continuum models using PSO and LMA optimization algorithms. A Drucker-Prager plasticity model with ductile damage is implemented in the CDM model to empirically capture the effect of the degradation (damage) of the NFR as deformation takes place.

The Drucker-Prager model with ductile damage is shown to be a reasonable CDM model approach to represent NFR in a continuum context including effects of pressure dependent yield and the triaxiality based damage initiation criterion. Compared to a full DEM simulation, the CDM model shows a good fit pre-damage, but is unable to emulate the subtle post-yield oscillations arising from non-continuous yielding in the NFR.

Most importantly, with this up-scaling framework, very comparable results ($<5\%$ error) to full DEM solutions can be obtained with the CDM method but with two to three orders of magnitude less computational time.

The implementation of this up-scaling framework is presented in an illustrative context only. This specific formulation is formulated only for small strains, it is a purely mechanical formulation (no thermo-hydro-mechanical coupling), and only 2D is considered. However, the proposed framework can accommodate the large strain case, the hydro-mechanically coupled case, and the 3-D case by using appropriately designed modules. We suspect that in these more complex cases, the up-scaling from DEM to CDM will yield even greater increases in computational efficiency. 











\clearpage