In addition to the hardening rules, the damage evolution equations must also be parameterized. The compressive damage, $D_c$, is assumed to be a linear function of the inelastic strain through a compressive damage rate parameter, $m$:

\begin{equation}
D_{c}\left(\bar{\epsilon}^{in}\right)=\bar{\epsilon}^{in}m\label{eqn:param3}
\end{equation}


The tensile damage ($D_{t}$) evolution is slightly less trivial,
but can also be characterized by a single parameter due to some constraints
imposed on the function by the nature of the damage parameter. In
tension, the damage evolution curve starts at the origin and asymptotically
approaches $D_{t}=1$ as $\bar{\epsilon}^{ck}\rightarrow\infty$.
As such, under this functional assumption, the only parameter required
to describe this relationship is the tensile damage rate parameter,
n:

\begin{equation}
D_{t}\left(\bar{\epsilon}^{ck}\right)=1-\frac{1}{\left(1+\bar{\epsilon}^{ck}\right)^{n}}\label{eqn:param4}
\end{equation}


Sample damage evolution curves for both tension and compression are
illustrated in Fig \ref{fig:concdam}, where one can see that the rate at which the
tensile damage evolves is far larger than the rate at which the compressive
damage evolves. The combination of the elastic parameters, the hardening
rule parameters, and the damage evolution parameters, yield a total
of nine parameters that must be identified by experiments or through
up-scaling to define the behavior of CDM model.

