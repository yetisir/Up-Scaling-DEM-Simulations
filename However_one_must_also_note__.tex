In this homogenization method, the aim is to obtain the spatially averaged stress and strain tensors at any prescribed time. An important aspect of this homogenization method is in the treatment of discontinuities between the blocks. In the case where the blocks become physically separated, there exists voids within the homogenization domain and discontinuities along the homogenization boundary as can be seen in Figure \ref{fig:homoboundary}. These discontinuities along the homogenization boundary are important to consider as most of the strain in the model is manifested in theses fractures. To account for these boundary discontinuities, discontinuity boundary segments are added to the homogenization boundary between the corners of the adjacent blocks to account for this strain. Within the homogenization domain, it is important to account for the voids as they are assumed to have a zero stress state, which becomes influential when assessing the homogenized stress tensor.

