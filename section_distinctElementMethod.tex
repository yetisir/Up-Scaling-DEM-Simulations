\section{Distinct/Discrete Element Method}
Discontinuous systems are characterized by the existence of discontinuities that separate discrete domains within the system. In order to effectively model a discontinuous system, it becomes necessary to represent two distinct types of mechanical behaviour: the behaviour of the discontinuities and the behaviour of the solid material.

There exists a set of methods, referred to as Discrete Element Methods, which provide the capacity to explicitly represent the behaviour of multiple intersecting discontinuities. The methods allow for the modelling of finite displacements and rotations of discrete bodies, including contact detachment as well as automatic detection of new contacts. Within the set of Discrete Element Methods, \citet{CUNDALL_1992} describes four subsets: Modal Methods, Discontinuous Deformation Analysis Methods, Momentum Exchange Methods, and Distinct Element Methods (DEM).


With DEM methods, the discontinuous system is represented as an assembly of deformable blocks such that the interfaces between the blocks represent the discontinuities. With respect to NFR, the blocks can be used to represent the intact rock while the discontinuities represent the joints in the rock mass. 

In this paper, ITASCA\textsuperscript{TM}'s implementation of a 2-dimensional DEM code, UDEC\textsuperscript{TM} was used for all the DEM simulations. With UDEC\textsuperscript{TM}, the elements of the discretized DEM mesh are referred to as blocks, which are discretized using a Finite Difference Method (FDM). The elements associated with this mesh are referred to as zones. The nodes that define the FDM zones are referred to as grid points, and the grid points that exist on the edge of the blocks are a special kind of gripoint referred to as corners. When two blocks interact with each other, a contact is defined which governs the block interaction behaviour.  Figure \ref{fig:DEM} shows schematically the relation of these key components and some of the contact physics. The contact behaviour here is described primarily by shear stiffness, $k_s$, and normal stiffness, $k_n$, which hold the blocks together until the tensile stress in the contact exceeds the contact tensile strength. 

Consider an arbitrary deformable domain, $\Omega$, with a boundary, $\Gamma$, that is subdivided by discontinuities into $i$ number of deformable subdomains, each denoted by $\Omega_i$ (Figure\ref{fig:DEM}). Let $\Gamma_{ij}=\Gamma_{ji}$ represent the boundary between $\Omega_i$ and $\Omega_j$. The motion of these subdomains is governed by the conservation of momentum which relates the stress field divergence, $\nabla\boldsymbol{\sigma}^m$, to the domain acceleration, $\ddot{\mathbf{u}}^m$ and density, $\rho$:
