\subsection{Impact of REV Size on Estimated Parameters}

The appropriateness of the REV size was tested using eight different sample REV radii and running the homogenization and parameter estimation algorithms for each. The assumed REV radius for the parameter estimation simulations is $4m$, which corresponds to a homogenization area of $57.7 m^2$. To validate this assumption, the REV radii is sampled at $0.5m$ intervals to see where the resultant parameters converge.

The convergence of three of the 11 parameters is shown in Figure \ref{fig:revconverge} as a function of REV size. The material parameters apparently converge at different sizes, illustrating part of the challenge in defining an REV; some parameters require a larger REV than others and it is not obvious \textit{a priori} which parameters will dominate. For the granite rock mass considered, an REV of radius $3m$ or with a homogenization area of $34.8 m^2$ is chosen to be the minimum size based on the convergence of the dilation angle - the last parameter to converge. The suitability of the assumed REV size is confirmed since it is larger than the minimum REV size determined by the convergence study.