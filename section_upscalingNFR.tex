\section{Verification}

In this section, two-dimensional DEM models are used to demonstrate the effectiveness of the up-scaling methodology. While the results presented leave a positive impression, future verification using three-dimensional models is desirable. The framework is validated using three tests: 1) The homogenized stress and strain behaviour obtained from the DEM microscale response are compared to that of the macroscale response. This test verifies the effectiveness of the parameter estimation module and the ability of the chosen macroscale constitutive model to capture the salient features of NFR behaviour. 2) The homogenization and parameter estimation algorithms are rerun using the same data, but with different REV sizes to investigate the REV size effect has on the resultant parameter set. 3) Slope stability analyses carried out by both Direct Numerical Simulations (DNS) with a DEM model and with an up-scaled macroscale model are compared. This last test verifies the whole up-scaling methodology.

The up-scaling is conducted by running a series of four quasi-static DEM 'triaxial' compression tests under different confining stresses. These are not true triaxial tests as simulations are in 2D, but illustrate the method regardless. Algorithms are rerun using different REV sizes to determine an appropriate REV size. In a macroscopically homogenous domain, as the REV size increases, the parameter values will converge to a single value, when the REV is too small, local heterogeneities induce a variance into the optimal parameter set.

