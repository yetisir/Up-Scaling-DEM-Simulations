\section{Verification}

In this section, several studies are presented which aim to demonstrate the effectiveness of the up-scaling methodology. Two-dimensional DEM models are used for this purpose. While the results presented leave a positive impression, further verification using three-dimensional models is desirable in the future. The framework is validated using three tests: 1) The homogenized stress and strain behaviour obtained from the DEM microscale response are compared to that of the macroscale response with an near-optimum parameter. This test verifies the effectiveness of the parameter estimation module. 2) The homogenization and parameter estimation algorithms were rerun using the same date as in the first test, but with different REV sizes to investigate the effects the REV size has on the resultant parameter set. 3) Slope stability analyses  carried out by both Direct Numerical Simulations (DNS) with a DEM model and with an up-scaled macroscale model are compared. This last test verifies the whole up-scaling methodology.

 The up-scaling was conducted by running a series of four quasi-static DEM 'triaxial' compression tests under different confining stresses. These were not true triaxial tests due to the simulations being performed in 2D, but illustrate the method regardless.

The axial stress-strain data was taken to be the most indicative of the material response and thus used for the parameter estimation datasets. Using the aforementioned homogenization and parameter estimation algorithms, an optimal parameter set was able to be determined from loading tests.

The homogenization and parameter estimation algorithms were rerun using different REV sizes to investigate the effects the REV size has on the resultant parameter set. In general, in a macroscopically homogenous domain, as the REV size increases, the parameter values will converge to a single value, when the REV is too small, the local heterogenaities will induce a variance into the optimal parameter set. From this REV investigation, an appropriate REV size was able to be assessed and applied. 

