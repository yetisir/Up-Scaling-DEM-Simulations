\subsection{Drucker-Prager Plasticty Model With Ductile Damage}

the Drucker-Prager plasticity model was developed by by \citet{drucker_implications_1950} for modeling frictional materials like granular soils and rock. These materials tend  to exhibit pressure dependent yielding (as confining
pressure increases, so does the strength of the rock) which was incorporated as a pressure dependent yield criterion in the Drucker-Prager plasticity model. 

Specifically, the Drucker-Prager material model is formulated and used for materials of which the compressive yield strength is much greater than the tensile yield strength such as one would find in soils and rocks. Howewver, one drawback with this material model is that it is intended to simulate material response under essentially monotonic loading which limits the potential of modeling cyclic loading tonight.

In addition, the Drucker-Prager model is suitable for using in conjunction with progressive damage and failure models. In this formulation, the Johnson-Cook Damage model is used to model the ductile damage of the rock mass \cite{johnson_fracture_1985}. At a sufficiently large scale, the damage behaviour of NFR can be thought of as behaving in a ductile capacity. 

%The linear Drucker-Prager yield function, which defines the pressure
%dependent behaviour of the material can be written as follows:

%\begin{equation}
%F=t-p\tan\left(\beta\right)-d\label{eqn:druc1}
%\end{equation}


In this paper, the hyperbolic Drucker-Prager yeild function was chosen to represent the material yeild behaviour and is expressed as a function of the mises equivalent stress and hydrostatic stress:

\begin{equation}
F\left(p,q\right)=\sqrt{l_{0}^{2}+q^{2}}-p\tan\left(\beta\right)-d'\label{eqn:druc2}
\end{equation}

Where $\beta$ is the friction angle, and d is a hardening parameter defined as a function of the uniaxial compressive yield stress, $\sigma_c$:

\begin{equation}
d'=\sqrt{l_{0}^{2}+q^{2}}-frac{\sigma_c}{3}\tan\left(\beta\right)
\label{eqn:druc2-2}
\end{equation}

Furthermore, $l_0$ is a variable introduced for simplicity that helps describes the behaviour of the tensile portion of the yield function: 

\begin{equation}
l_{o}=d'_{0}-p_{0}^{t}\tan\left(\beta\right)\label{eqn:druc2-1}
\end{equation}

This hyperbolic yield criterion is a combination of Rankine's maximum tensile stress condition at low confining stress and the linear Drucker-Prager condition at high confining stress. In the deviatoric stress plane, a von Mises section is used and and a hyperbolic flow potential in the meridontal plane. This hyperbolic model assumes a linear dependence between deviatoric stress and hydrostatic stress at high confining stresses, but allows for a non-linear relationship at low confining stress to allow for more accurate tensile yield behaviour.

%Flow Rule, same as concrete damage. should bring to main d-p material
%description:

%\begin{equation}
%G=\sqrt{\left[\epsilon\bar{\sigma}_{0}\tan\left(\psi\right)\right]^{2}+q^{2}}-p\tan\left(\psi\right)\label{eqn:druc5}
%\end{equation}


%default value of eccentricity: maybe

%\begin{equation}
%\epsilon=\label{eqn:druc5-1}
%\end{equation}


Hardening Rule -> not sure best way to write this

\begin{equation}
h=h\left(d',\bar{\sigma}\right)\label{eqn:druc6}
\end{equation}


where:

\begin{equation}
d'=\sqrt{l_{0}^{2}+\sigma_{c}^{2}}-\frac{\sigma_{c}}{3}\tan\left(\beta\right)\label{eqn:druc6-1}
\end{equation}


isotropic hardening is assumed, treating friction angle constant wrt
stress

{*}Damage initiation

- assume ductile damage. at large confining pressures at large scale,
the behaviour of rock is more ductile than brittle.

- onset of damage due to `` nucleation, growth, and coalescnce of
voids''. Model assumes PEEQ when damage is initiated is a function
of triaxiality, eta.

assume the form of the Johnson-Cook Model:

\begin{equation}
\bar{\epsilon}_{f}^{pl}\left(\eta,\dot{\bar{\epsilon}}^{pl},\hat{T}\right)=\left[D_{1}+D_{2}e^{D_{3}\eta}\right]\left[1+D_{4}\ln\left(\frac{\dot{\bar{\epsilon}}^{pl}}{\dot{\bar{\epsilon}}}\right)\right]\left[1+D_{5}\hat{T}\right]\label{eqn:druc7}
\end{equation}


assuming isothermal conditions and neglecting strain rate effects
because of reasons:

\begin{equation}
\bar{\epsilon}_{f}^{pl}\left(\eta\right)=\left[D_{1}+D_{2}e^{D_{3}\eta}\right]\label{eqn:druc8}
\end{equation}


{*}Damage evolution:

- assumes damage is a progressive degredation of the material stiffness

- assume isotropic damage

- uses mesh independant measure of plastic dispalcement to drive evolution
of damage

- damage manifests itself in two forms: softenin of the yeild stress
and degredation of the elastisity.

Damage evolution equation based on effective plastic displacement:

\begin{equation}
\dot{\bar{u}}^{pl}=L\dot{\bar{\epsilon}}^{pl}\label{eqn:druc9}
\end{equation}


asuming a linear form, the plastic displacement at at complete damage
($D=1$) can be specified, and the damage evolution can be written
as:

\begin{equation}
\dot{d}=\frac{L\dot{\bar{\epsilon}}^{pl}}{\dot{\bar{u}}_{f}^{pl}}=\frac{\dot{\bar{u}}^{pl}}{\dot{\bar{u}}_{f}^{pl}}\label{eqn:druc9-1}
\end{equation}
