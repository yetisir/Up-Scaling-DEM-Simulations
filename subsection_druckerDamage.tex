


%The linear Drucker-Prager yield function, which defines the pressure
%dependent behaviour of the material can be written as follows:

%\begin{equation}
%F=t-p\tan\left(\beta\right)-d\label{eqn:druc1}
%\end{equation}


In this paper, the hyperbolic Drucker-Prager yeild function was chosen to represent the material yeild behaviour and is expressed as a function of the mises equivalent stress and hydrostatic stress:

\begin{equation}
F\left(\bar{\sigma}_{ij}, \bar{\epsilon}^{pl}\right)=\frac{1}{2}q\left(\bar{\sigma}_{ij}\right)\left [ 1+\frac{1}{K}-\left ( 1-\frac{1}{K} \right )\left ( \frac{r\left(\bar{\sigma}_{ij}\right)}{q\left(\bar{\sigma}_{ij}\right)} \right )^3 \right ]-p\left(\bar{\sigma}_{ij}\right)\tan\beta-d\label{eqn:druc2}
\end{equation}

Where $\beta$ is the friction angle, and $d$ is a hardening parameter defined as a function of the uniaxial compressive yield stress, $\sigma_c$:

\begin{equation}
d'\left(\bar{\sigma}_{ij}, \bar{\epsilon}^{pl}\right)=\sqrt{l_{0}^{2}+q\left(\bar{\sigma}_{ij}\right)^{2}}-\frac{\sigma_c\left(\bar{\epsilon}^{pl}\right)}{3}\tan\left(\beta\right)
\label{eqn:druc2-2}
\end{equation}


