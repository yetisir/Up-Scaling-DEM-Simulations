\subsection{Verification of the Parameter Estimation Module}

Using a PSO algorithm followed by an LMA optimization, the Drucker-Prager plasticity model with ductile damage is then fitted to the homogenized DEM simulation data in order to obtain an optimal parameter set. Each simulation is fit to 50 points defining the homogenized stress-strain curve resulting in a total of 200 data points for all four DEM simulations at different confining stresses. The PSO algorithm uses a swarm size of 24 for 100 generations which is found to be sufficient to converge to a consistent solution. 

Here, the CDM model is confined laterally by the homogenized horizontal DEM stress and vertical displacements are prescribed by the homogenized vertical DEM strain with the parameter estimation algorithms programmed to match the horizontal strain and the veritcal stress. Because of the large variation in observation magnitudes (between stress/strain and from different confining stresses), each curve is weighted with a normalization factor to prevent the large stress values from dominating parameter estimation. In addition, a linear weighting scheme is applied to each curve to give larger influence to the loading section and lesser influence to the post-damage section.

Parameter bounding limits are required by the optimization algorithms in order to limit the search space in these optimization algorithms. These limits are chosen based on two criteria: physical limitations and numerical stability. If there exist physical limitations that prevent parameters from exceeding certain values or if there exists a range of realistic values that the parameter should not deviate from, then those physical limitations are specified as the bounds. In other cases, the parameter bounds come from numerical limitations such that beyond a certain capacity, certain parameter values would cause the simulations to become unstable. In these cases, a combination of the two bounding methods is used. The specified bounding limits for each parameter results can be seen in Table \ref{tab:paramDrucker}.

\begin{table}[!htbp]
\centering
\caption{Parameter Estimation Results for Drucker-Prager Model with Ductile Damage}
\label{tab:paramDrucker}
\begin{tabular}{@{}lccccc@{}}
\toprule
\multicolumn{1}{c}{\textbf{Parameter}} & \textbf{Symbol}                  & \textbf{Units} & \textbf{\begin{tabular}[c]{@{}c@{}}Lower \\ Bound\end{tabular}} & \textbf{\begin{tabular}[c]{@{}c@{}}Upper \\ Bound\end{tabular}} & \textbf{\begin{tabular}[c]{@{}c@{}}Optimal \\ Value\end{tabular}} \\ \midrule
Young's Modulus                        & $E$                              & $GPa$          & $1$                                                             & $25$                                                            & $1.8$                                                             \\
Poisson's Ratio                        & $\nu$                            &                & $0.1$                                                           & $0.4$                                                           & $0.15$                                                            \\
Dilation Angle                         & $\psi$                           & $^{\circ}$     & $5$                                                             & $15$                                                            & $22$                                                              \\
Flow Stress Ratio                      & $K$                              &                & $0.78$                                                          & $1$                                                             & $0.81$                                                            \\
Friction Angle                         & $\beta$                          & $^{\circ}$     & $45$                                                            & $60$                                                            & $56$                                                              \\
Initial Compressive Yield Strength     & $\sigma_c^{iy}$                  & $kPa$          & $1$                                                             & $100$                                                           & $52$                                                              \\
Peak Compressive Yield Strength        & $\sigma_c^{p}$                   & $MPa$          & $0.5$                                                           & $5$                                                             & $3.1$                                                             \\
Strain at Peak Compressive Yield       & $\epsilon_c^{p}$                 & $\%$           & $0.5$                                                           & $5$                                                             & $1.7$                                                             \\
Yield Strain at -0.5 Triaxiality       & $\bar{\epsilon}^{pl}_{f_{-0.5}}$ & $\%$           & $0.01$                                                          & $0.1$                                                           & $0.0078$                                                          \\
Yield Strain at -0.6 Triaxiality       & $\bar{\epsilon}^{pl}_{f_{-0.6}}$ & $\%$           & $0.1$                                                           & $10$                                                            & $0.30$                                                            \\
Plastic Displacement at Failure        & $\bar{u}^{pl}_f$                 & $m$            & $0.01$                                                          & $1$                                                             & $0.12$                                                            \\ \bottomrule
\end{tabular}
\end{table}

The stress-strain curves from the DEM simulations used for the parameter estimation and the stress-strain curves of the CDM simulations using the optimal parameter set are presented in Figure \ref{fig:fitted1}. The CDM fit is good with a Root-Mean-Square Error (RMSE) of $1.03MPa$ and the pressure dependent yield function works well with this model as the error is not biased to curves of a certain confining stress. This fit implies a strong likelihood that the model will be valid under confining stresses outside of the range fitted. Also, the damage initiation points at the peak of the curve are well correlated and indicate that the triaxiality based damage initiation criterion is a good model for this problem. The majority of the error in the curves is found in the post-yield behaviour. This error results from limitations in the continuum constitutive model because the post-yield behaviour of the DEM simulations is discontinuous in nature (stick-slip response). The CDM model cannot accommodate for such oscillations and thus represents the post-yield response as an average. 
