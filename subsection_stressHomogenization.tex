\subsection{Stress Homogenization}

The homogenized Cauchy stress, $\langle\boldsymbol{\sigma}\rangle$,
is derived from the definition of the spatial average of the stress,
$\boldsymbol{\sigma}$, over the homogenization domain $\Omega^{h}$,
with an area $A^{h}$: 
\begin{equation}
\langle\boldsymbol{\sigma}\rangle=\frac{1}{A^{h}}\int_{\Omega^{h}}\boldsymbol{\sigma}dA\label{eqn:stress1}
\end{equation}


As previously mentioned, there exists a necessary distinction between
the block subdomain and the void subdomain when dealing with stress.
Here, the integration over the homogenization domain can be decomposed
into two seperate integrations over the block subdomain and the void
subdomain: 
\begin{equation}
\langle\boldsymbol{\sigma}\rangle=\frac{1}{A^{h}}\left[\int_{\Omega^{b}}\boldsymbol{\sigma}dA+\int_{\Omega^{v}}\boldsymbol{\sigma}dA\right]\label{eqn:stress2}
\end{equation}


This distinction is made due to the fact that in a purely mechanical
model, there isn't any pressure or stress being retained in the void
space, resulting in a negligable contribution to the overall stress
state. This allows for the void subdomain integration term to be dropped
from the formulation: 
\begin{equation}
\langle\boldsymbol{\sigma}\rangle=\frac{1}{A^{h}}\int_{\Omega^{b}}\boldsymbol{\sigma}dA\label{eqn:stress2a}
\end{equation}


Since the block subdomain is inherently discretized, the integration
of the stress over the block subdomain can be written as a summation
of the average stress in each block, $\boldsymbol{\sigma}_{i}^{b}$,
of area $A_{i}^{b}$over $N^{b}$ number of blocks in the block subdomain:
\begin{equation}
\langle\boldsymbol{\sigma}\rangle=\frac{1}{A^{h}}\sum_{i=1}^{N^{b}}\boldsymbol{\sigma}_{i}^{b}A_{i}^{b}\label{eqn:stress3}
\end{equation}


In the case where the blocks are considered as deformable bodies,
the average stress in a given block can be assessed by taking a spatially
weighted average of the zone stresses ($\boldsymbol{\sigma}_{ij}^{z}$).
For a given block with $N_{i}^{z}$ zones, the average stress can
be weighted based on the zone area, $A_{ij}^{z}$: 
\begin{equation}
\boldsymbol{\sigma}_{i}^{b}=\frac{1}{A_{i}^{b}}\sum_{j=1}^{N_{i}^{z}}\boldsymbol{\sigma}_{ij}^{z}A_{ij}^{z}\label{eqn:stress3a}
\end{equation}


Combining equations \ref{eqn:stress3} and \ref{eqn:stress3a} yield
the final form for assessing the homogenized Cauchy stress tensor
for deformable DEM blocks: 
\begin{equation}
\langle\boldsymbol{\sigma}\rangle=\frac{1}{A^{h}}\sum_{i=1}^{N^{b}}\sum_{j=1}^{N_{i}^{z}}\boldsymbol{\sigma}_{ij}^{z}A_{ij}^{z}\label{eqn:stress4}
\end{equation}


%In the case that the blocks are considered as rigid bodies, the determination of the block stresses is less trivial. Becasue a rigid body cannot undergo deformation, and therfore not experience any strain, the concept of stress in a undeformable block is unreasonable. In this case, the stress homogenization procedure focuses on reducing the block boundary contact forces to a resultant force-moment pair acting on the center of mass of the block. 

