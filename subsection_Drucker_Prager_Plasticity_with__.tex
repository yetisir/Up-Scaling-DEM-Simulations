\subsection{Drucker-Prager Plasticity with Ductile Damage}

The Drucker-Prager plasticity model can be written in terms of 4 individual
components: The friction angle, dilation angle, initial tensile strength,
and the hardening rule. The hardening rule assumes a functional form
in terms of a power law according to \cite{prantl_identification_2013}:

\begin{equation}
\sigma_{y}=A+B\left(\bar{\epsilon}^{pl}\right)^{n}\label{eqn:dparam5}
\end{equation}

This yields a total of 6 parameters for the plasticity characterization.
The damage characterization is comprised of two main components, the
damage initiation and the damage evolution. The damage initiation
is based on the Johnson-Cook model as follows (from before):

\begin{equation}
\bar{\epsilon}_{f}^{pl}\left(\eta\right)=D_{1}+D_{2}e^{D_{3}\eta}\label{eqn:dparam6}
\end{equation}


To minimize the number of parameters that need to be estimated, the
above is simplified to the following by neglecting the first Johnson-Cook
Constant:

\begin{equation}
\bar{\epsilon}_{f}^{pl}\left(\eta\right)=D_{2}e^{D_{3}\eta}\label{eqn:dparam6-1}
\end{equation}
