The optimal parameter set in Table \ref{tab:paramDrucker} represents the constitutive response of the rock mass. As expected, the elastic modulus of the rock mass ($1.9 GPa$) is substantially less than the elastic modulus of the intact rock ($65 GPA$) because of yielding in the joints. Additionally, Poisson's ratio of the rock mass ($0.15$) is less than Poisson's ratio of the intact rock ($0.2$) because of the compliance of the joints before yield, which limits the lateral strain. After yielding however, substantial lateral strain is observed because of dilation of the joints resulting in a large dilation angle ($22^\circ$). This dilation response of the rock mass is larger than the the prescribed joint dilation ($5^\circ$) because of block rotation and geometry.

There are additional minor sources of error from the homogenization algorithms that do not manifest themselves in this fitted relationship.  In addition, if the REV is too small, it introduces it's error in the DEM data rather than in the fitted response. Furthermore, the global fitting algorithms are not completely exhaustive, so it is possible they do not find the actual globally optimal parameter set, potentially leading to some error. With the given PSO parameters, up to 2400 sets of simulations are conducted for the global parameter estimation, and successive fitting operations tend to give results within $1\%$ deviation. This consistency and large search domain give confidence that the estimated parameter set is the globally optimal set. 

In addition to the loading response under the specified confining stresses, DEM simulations under confining stresses of $3MPa$, $6MPa$, $8MPa$ and $10MPa$ are compared to the the CDM model using the previously estimated parameter set to see how well the constitutive behaviour is captured (Figure \ref{fig:fitted2}). These simulations demonstrate the interpolative ($3MPa$) and extrapolative ($6MPa$, $8MPa$ and $10MPa$) capacity of the fitted parameter set, and indeed a strong fit is obtained (RMSE of $2.83MPa$) for all confining stresses, with the error being more prominent for larger degrees of strain.