\section{ModelParameterization}
parameterization of the material models is essential to effective upscaling of the system. parameterization needs to be conducted in susch a way as to minimize the number of parameters, while still yeilding a sufficiently accurate response. in addition, it is ideal, but not necessary for the parameters to have physical meaning in order to specify realistic bounds with less difficulty.

A total of nine parameters were identified as being essential to the definition of the CDM material model. The elastic behavior of the rock mass was described using two parameters, Young’s Modulus, and Poisson’s Ratio, while 5 additional parameters were used to describe the plastic behavior of the rock mass in both tension and compression. The remaining two parameters describe the damage evolution in tension and compression respectively.

This investigation uses an implementation by \citet{matott_ostrich:_2008} in the form of model-independent optimization software, OSTRICH.

In order to accurately assess the optimal parameter set, the parameter estimation of the CDM material model was conducted in two steps. The first step consisted of a uniaxial tension test to estimate the elastic and tensile plastic properties of the material. The test direction reverses to compression before complete yield of the material in order to assess the tensile damage parameter. 

