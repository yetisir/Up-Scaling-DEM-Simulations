\subsection{Model Parameterization}

The CDM model used in this investigation is the concrete damaged plasticity model that is implemented in ABAQUS, which is based on the plastic-damage model for concrete proposed by Lubliner et al. (1989) and further developed by Lee and Fenves (1998) for cyclic loading. The general CDM theory considers the stiffness degradation of the material by modifying the elastic stiffness tensor with a damage variable. The damage variable can be scalar or tensorial in nature, depending on the anisotropy of damage evolution. In this investigation, the isotropic case is considered and so a scalar damage variable becomes sufficient.
The parameters of the CDM that were calibrated from DEM simulations characterize the elastic and the plastic behavior and the damage evolution. Young’s modulus and Poisson’s ratio parameterize the elastic behavior. The plastic behavior of the material is largely determined by the parameters of the hardening rule, and different functions were used to characterize hardening in tension and compression. The shape of the curves were based on the laboratory data in Wahalathantri et al. (2011) and the aim was to mimic the curves with the least number of parameters. 
The compressive hardening rule, σ_c (ε ̃^in ), was approximated using a quadratic function, as shown in Fig 3. The quadratic function requires three parameters. It was found to be useful to manipulate the standard quadratic equation form to allow for the three parameters to have a physical meaning. This form of the approximation becomes quite useful for the parameter estimation when applying bounding limits:
	σ_c (ε ̃^in )=(σ_c^iy-σ_c^p)/(ε_c^pp )^2  (ε ̃^in-ε_c^pp )^2+σ_c^p		
 
Fig 3	Compressive hardening rule for the CDM model using three parameters for a quadratic approximation.
Here, the compressive yield stress, σ_c, is written as  function of the inelastic strain, ε ̃^in, and three additional parameters. The three parameters are the initial compressive yield stress (σ_c^iy), the peak compressive yield stress (σ_c^p), and the plastic strain at the peak compressive yield stress (ε_c^pp). The physical significance of each of these parameters can be seen in Fig 3, where they define the y-intercept and the peak of the curve. 
The tensile hardening rule has a fundamentally different behavior than the compressive hardening rule, and was therefore approximated using an exponential function (Fig 4). The exponential function required only two parameters to characterize the curve completely. The first parameter was the initial tensile yield stress, σ_t^iy, which defines the y-intercept of the curve, while the second parameter was the tensile yield stress decay parameter, λ. These parameters describe the relationship between the tensile yield stress, σ_t, and the cracking strain, ε ̃^ck, and has the form:
	σ_t (ε ̃^ck )=σ_t^iy e^(λε ̃^ck )		
 
Fig 4	Tensile hardening rule for the CDM model using two parameters for an exponential approximation.
In addition to the hardening rules, the damage evolution equations must also be parameterized. The compressive damage, D_c, is assumed to be a linear function of the inelastic strain through a compressive damage rate parameter, m:
	D_c (ε ̃^in )=ε ̃^in m		
The tensile damage (D_t) evolution is slightly less trivial, but can also be characterized by a single parameter due to some constraints imposed on the function by the nature of the damage parameter. In tension, the damage evolution curve starts at the origin and asymptotically approaches D_t=1 as ε ̃^ck→∞. As such, under this functional assumption, the only parameter required to describe this relationship is the tensile damage rate parameter, n:
	D_t (ε ̃^ck )=1-1/(1+ε ̃^ck )^n 		
Sample damage evolution curves for both tension and compression are illustrated in Fig 5, where one can see that the rate at which the tensile damage evolves is far larger than the rate at which the compressive damage evolves. 
The combination of the elastic parameters, the hardening rule parameters, and the damage evolution parameters, yield a total of nine parameters that must be identified by experiments or through up-scaling to define the behavior of CDM model. 
 
Fig 5	Damage evolution for both tension and compression for the CDM model using only one rate parameter.


