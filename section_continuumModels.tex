\section{Continuum Damage Mechanics Models}
The constitutive stress-strain relationship including a scalar damage parameter, $\mathbf{D}$ can be written as follows:
\begin{equation}
\label{eqn:const3}
\boldsymbol{\sigma} = (1-\mathbf{D})\mathbf{E}:\boldsymbol{\epsilon^{el}}
\end{equation}

For simplicity, the damaged elastic stiffness is described as the reduced stiffness due to the damage:
\begin{equation}
\label{eqn:const4}
\mathbf{E^d} = (1-\mathbf{D})\mathbf{E}
\end{equation}

Substituting \ref{eqn:const1} and \ref{eqn:const4} into \ref{eqn:const3} results in the following:
\begin{equation}
\label{eqn:const5}
\boldsymbol{\sigma} = \mathbf{E^d}:(\boldsymbol{\epsilon}-\boldsymbol{\epsilon}^{pl})
\end{equation}

Using the "usual notions of CDM" (find reference), the effective stress. $\boldsymbol{\bar{\sigma}}$, can be defined as:
\begin{equation}
\label{eqn:const6}
\boldsymbol{\bar{\sigma}} = \mathbf{E}:(\boldsymbol{\epsilon}-\boldsymbol{\epsilon}^{pl})
\end{equation}

Such that the cauchy stress tensor can be realted to the effective stress tensor as follows:
\begin{equation}
\label{eqn:const7}
\boldsymbol{\sigma} = (1-\mathbf{D})\boldsymbol{\bar{\sigma}}
\end{equation}

The nature of the damage evolution is assumed to be a function of the effective stress and the equivalent plastic strain, $\boldsymbol{\bar{\epsilon}^{pl}}$:
\begin{equation}
\label{eqn:const8}
\mathbf{D} = \mathbf{D}(\boldsymbol{\bar{\sigma}}, \boldsymbol{\bar{\epsilon}^{pl}})
\end{equation}